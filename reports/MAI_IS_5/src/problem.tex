\CWHeader{Лабораторная работа \textnumero 5 \enquote{Ранжирование TF-IDF}}

Необходимо сделать ранжированный поиск на основании схемы ранжирования TF-IDF. Теперь,
если запрос содержит в себе только термины через пробелы, то его надо трактовать как нечёткий
запрос, т.е. допускать неполное соответствие документа терминам запроса и т.п. Примеры
запросов:
\begin{itemize}
    \item \text{[} роза цветок ]
    \item \text{[} московский авиационный институт ]
\end{itemize}

Если запрос содержит в себе операторы булева поиска, то запрос надо трактовать как булев, т.е.
соответствие должно быть строгим, но порядок выдачи должен быть определён ранжированием
TF-IDF. Например:
\begin{itemize}
    \item \text{[} роза \&\& цветок ]
    \item \text{[} московский \&\& авиационный \&\& институт ]
\end{itemize}

В отчёте нужно привести несколько примеров выполнения запросов, как удачных, так и не
удачных.

\pagebreak
