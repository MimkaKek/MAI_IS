\section{Выводы}

Выполнив шестую лабораторную работу по курсу \enquote{Информационный поиск}, я :
\begin{itemize}
    \item убедился, что закон Ципфа скорее годится для случаев, когда речь идёт о случайном наборе слов;
    \item если речь идёт о таком наборе слов, в котором есть некоторая закономерность в виде тематики, то этот закон работает не так хорошо как хотелось бы;
    \item также этот закон довольно плохо соотносится с участками, где частоты малы или наоборот крупны;
    \item формула Мандельброта позволяет несколько точнее смоделировать участок с высокой частотой, однако как и закон Ципфа не в состоянии описать (по крайней мере в моём случае) участки средней и малой частоты;
    \item скорее всего, если бы речь шла о полностью случайном наборе слов, то вопрос с участком средней частоты бы отпал, однако эти две формулы вряд ли бы описали участок с малой частотой.
\end{itemize}
\pagebreak
