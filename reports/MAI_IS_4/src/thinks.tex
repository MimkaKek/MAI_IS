\section{Выводы}

Выполнив четвёртую лабораторную работу по курсу \enquote{Информационный поиск}, я :
\begin{itemize}
    \item выботал некоторые навыки для построения синтаксического дерева;
    \item запросы, в которых присутствует отрицание - самые проблемные в реализации. Не могу бы уверен, что есть смысл пытаться реализовать и его;
    \item список файлов является не самой лучшей идеей, когда речь идёт о запросах, которым соответствует огромное количество файлов. Первая мысль - как-то ограничить количество файлов (например оставить лишь первые 50 файлов после ранжировки), вторая мысль - попробовать найти иную структуру данных, которая стала бы хорошей заменой списку.
\end{itemize}
\pagebreak
