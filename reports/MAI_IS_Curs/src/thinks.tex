\section{Выводы}

Выполнив курсовую работу по курсу \enquote{Информационный поиск}, я :
\begin{itemize}
    \item осознал, что перевод наряду с лемматизацией занимает достаточно много времени;
    \item пришёл к выводу, что программу можно несколько ускорить за счёт записи всех токенов в один файл, а не в множество. Это позволить всё свести к чтению одного большого файла, что в теории должно дать прирост производительности. Правда в таком случае будет несколько сложнее по необходимости узнать какие токены принадлежат конкретному файлу. Но это можно подправить специальными метками;
    \item код можно написать в ряде мест иначе и лучше. Если бы это был полноценный проект, то я бы постарался довести его до ума и оптимизировать.
\end{itemize}
\pagebreak
